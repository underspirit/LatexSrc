\documentclass[10pt,a4paper,notitlepage,onecolumn]{article}
\usepackage[normalem]{ulem} %\underline 命令可以给字体加下划线,但是它不能正确断字。 ulem 宏包改进了断字,还增加了波浪线和删除线等命令。只是 ulem 把 \emph 重定义成了下划线,我们可以在引用宏包时可以加个选项改回去: \usepackage[normalem]{ulem}。
\begin{document}
        \section{section 1}
             section 1 content 
             \\ %换行
             \newline   %换行
             %\newpage   %换页
             %各种符号
             \^ \& \_\_ \{ \} \~ \textbackslash \% 
             -  %短划线
              \today 
              --    %中划线
               \TeX 
               ---  %长划线
                \LaTeX \\
            \textcircled{A} 
            %字体样式
            \\
            \tiny   %字体大小
            \textrm{subsection 1 content}\\ %roman
            \small
            \textbf{subsection 2 content}\\   %设置加粗
            \Large
            \textit{paragraph 1 content}\\    %斜体
            \textsf{subsubsection 1 content}\\    %sans serif
            %字体强调和下划线
            \emph{emphasis}\\
            \uline{underline}\\
            \underline{underline}\\ %自带
            \uwave{waveline}\\
            \sout{strike -out}\\
            \hyphenation{BASIC blar-blar-blar}
\end{document}
