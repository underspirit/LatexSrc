\documentclass[12pt]{article}
\usepackage{amsmath}    %使用数学功能

\begin{document}
    \section{section 1}
    %希腊字母
    $\alpha$ \ $\beta$ \ $\gamma$ \\    %小写
    $\Delta$ \ $\Theta$ \ $\Gamma$ \\   %大写
    
    %上下标和根号
    %指数或上标用 ^ 表示,下标用 _ 表示,根号用 \sqrt 表示。上下标如果多于一个字母或符号,需要用一对 {} 括起来。
    \[ x_{ij}^k \ \sqrt{x^{tk}} \ \sqrt[3]{x} \]
    
    %分数
    %分数用 \frac 命令表示,它会根据环境自动调整字号,比如在行间公式中小一点,在独立公式中则大一点。我们可以人工设置分数字号,比如\dfrac 命令把分数的字号设置为独立公式中的大小,而 \tfrac 命令则把字号设为行间公式中的大小。
    $ \frac{x^2}{y^2} \dfrac{x^2}{y^2} $
    \[ \frac{x^2}{y^2}
        \tfrac{x^2}{y^2} \]
        
    %运算符
    %基本符号
    \[ \pm \ \times \ \div \ \cdot \ \cap \ \cup \
        \geq \ \leq \ \neq \ \approx \ \equiv \]
    %高级一点
    %上下标方式,在行间公式中被压缩了
    $ \sum_{i=1}^n i \        \prod_{i=1}^n i \  \lim_{x\to0}x^2 \ \int_a^b x^2 dx  $ \\
    %limits功能,指定不压缩上小标
    $ \sum\limits_{i=1}^n i \ \prod\limits_{i=1}^n i \ \lim\limits_{x\to0}x^2 \ \int\limits_a^b x^2 dx $
    %上小标在独立公式中不被压缩
    \[ \sum_{i=1}^n i \        \prod_{i=1}^n i \  \lim_{x\to0}x^2 \ \int_a^b x^2 dx  \]
    %\nolimits指定在独立公式中压缩上下标
     \[\sum\nolimits_{i=1}^n i \ \prod\nolimits_{i=1}^n i \ \lim\nolimits_{x\to0}x^2 \ \int\nolimits_a^b x^2 dx \]
    %改变积分变量样式
    \newcommand{\myd}{\;\mathrm{d}}
    \[ \int x dx \ \int x \myd x \]
    %多重积分
    \[ \int\int \ \int\int\int \ \int\int\int\int \ \int\dots\int \]    %错误方法
    \[ \iint \ \iiint \ \iiiint \ \idotsint \]  %正确方法
    
    %箭头
    \[ \leftarrow \ \longleftarrow \ \rightarrow \ \longrightarrow \ \leftrightarrow \ \longleftrightarrow \]
    \[ \Leftarrow \ \Longleftarrow \ \Rightarrow \ \Longrightarrow \ \Leftrightarrow \ \Longleftrightarrow \]
    %可扩展箭头,箭头可以根据内容自动调整长度
    \[ \xleftarrow{x+y+z} \ \xrightarrow[x<y]{a*b*c} \]
    
    %注音和标注
    \[ \bar{x} \ \hat{x} \ \vec{x} \ \dot{x} \]
    \[ \overline{xyz} \ \underline{xyz} \ \overleftrightarrow{xyz} \ \overbrace{xyz} \]
    
    %分隔符,括号
    \[ \Bigg(\bigg(\Big(\big((x)\big)\Big)\bigg)\Bigg)\quad
    \Bigg[\bigg[\Big[\big[[x]\big]\Big]\bigg]\Bigg]\quad
    \Bigg\{\bigg\{\Big\{\big\{\{x\}\big\}\Big\}\bigg\}\Bigg\}
    \]
    \[\Bigg\langle\bigg\langle\Big\langle\big\langle\langle x
    \rangle\big\rangle\Big\rangle\bigg\rangle\Bigg\rangle\quad
    \Bigg\lvert\bigg\lvert\Big\lvert\big\lvert\lvert x
    \rvert\big\rvert\Big\rvert\bigg\rvert\Bigg\rvert\quad
    \Bigg\lVert\bigg\lVert\Big\lVert\big\lVert\lVert x
    \rVert\big\rVert\Big\rVert\bigg\rVert\Bigg\rVert 
    \]
    
    %省略号,\dots 和 \cdots的纵向位置不同;前者一般用于有下标的序列。
    \[ x_1,x_2,\dots,x_n \ 1,2\cdots,n \ \vdots \ \ddots \]
    %
    %空白间距
     a\,b \\ a\:b \\ a\;b \\ a\quad b \\ a\qquad b \\ a\!b 
    
\end{document}
