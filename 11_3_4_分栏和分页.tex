\documentclass[12pt]{article}
\usepackage[CJKchecksingle,CJKnumber]{xeCJK}
\setCJKmainfont[BoldFont={Adobe Heiti Std},ItalicFont={Adobe Kaiti Std}]{Adobe Song Std}
\setCJKsansfont{Adobe Heiti Std}
\setCJKmonofont{Adobe Fangsong Std}
\punctstyle{hangmobanjiao}

\usepackage{multicol}
\begin{document}
%简单的两栏设置
%\documentclass[twocolumn]{article}

%multicol 宏包提供了更多的功能,比如它支持多达十个栏
%位,栏位数目可以任意切换;各栏长度相同,最后一页看起来会左右平衡
%一些。
\setlength{\columnsep}{12pt}    %将栏位之间的距离设为 12pt(缺省是 10pt)
\setlength{\columnseprule}{1pt} %将栏位之间的分割线宽度设为 1pt(缺省 0pt,也就是不显示)
\begin{multicols}{4}
在浏览网页过程中,所有的“大”“小”概念,都是基于“屏幕”这个“界面”上。“屏幕”上的各种信息,包括文字、图片、表格等等,都会随屏幕的分辨率变化而变化,一个100px宽度大小的图片,在800×600分辨率下,要占屏幕宽度的1/8,但在1024×768下,则只占约1/10。所以如果在定义字体大小时,使用px作为单位,那一旦用户改变显示器分辨率从800到1024,用户实际看到的文字就要变“小”(自然长度单位),甚至会看不清,影响浏览。\\
那是不是用pt做单位就没这样的问题呢?错!问题同样出现。刚才的例子已经很清楚的说明,在不同分辨率下,无论是px还是pt,都会改变大小。以现在的电脑屏幕情况,还没有一种单位可以保证,在不同分辨率下,一个文字大小可以“固定不变”。因为这很难以实现也不是很有必要:全球电脑用户以亿来数,屏幕从14寸到40寸甚至更高都有,屏幕大小不同,分辨率也不同,要保证一个字体在所有用户面前大小一样,实在是MISSION IMPOSSIBLE。另外,电脑有其自身的调节性。

那在页面设计中到底是用px还是pt呢?

我认为,这个并没有什么原则性差异,就看自己处于什么角度思考了。
\end{multicols}

%分页
%TEX 通常都会自动分页,无须人工干涉。但是浮动体较多的情况下,分
%页就变成一个 NP 完全问题 5,自动分页的效果可能不是我们想要的。这时
%就需要手工插入分页命令,
\newpage
%如果我们也不确定某处分页是否妥当,可以使用另一个命令,给 TEX 留
%点面子。这个参数取值 1–4, 4 表示强烈要求分页, 1 表示你看着办吧。
\pagebreak[3]
%类似地,我们还可以建议 TEX 不要分页,其参数取值也是 1–4, 4 表示
%强烈反对分页, 1 表示随便。
\nopagebreak[2]
%浮动体较多, TEX 无所适从时,我们可以用下面的命令帮它减轻点责
%任。此命令要求 TEX 排完此前所有浮动体,不管是否难看,咱就这么办了。
\clearpage

\end{document}