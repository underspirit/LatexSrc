\documentclass[12pt,twoside]{book}
\usepackage[CJKchecksingle,CJKnumber]{xeCJK}
\setCJKmainfont[BoldFont={Adobe Heiti Std},ItalicFont={Adobe Kaiti Std}]{Adobe Song Std}
\setCJKsansfont{Adobe Heiti Std}
\setCJKmonofont{Adobe Fangsong Std}
\punctstyle{hangmobanjiao}

%改动 L ATEX 的页面布局缺省设置。当有特殊需要时,可以使用 2.4 节提到的 \setlength 或 \addtolength 来设置上述宏变量的值
%设置页面尺寸和边距,geometry提供了更高级的用户接口
%\usepackage[paperwidth=100mm, paperheight=150mm, margin=20mm]{geometry}
%单独设置每个边距,landscape把页面横来
\usepackage[top=2in, bottom=1in, left=1in, right=1in]{geometry}
\usepackage{fancyhdr}
\begin{document}
\renewcommand{\chaptername}{第\thechapter 章}
\chapter{发的}
\section{sdfg}
%自定义样式
\makeatletter %暂时把@它当正常符号用
\newcommand{\ps@permanentdamagedhead}{
\renewcommand{\@oddhead}{信春哥\hfill 不挂科}
\renewcommand{\@oddfoot}{\hfill\thepage\hfill}
\renewcommand{\@evenhead}{芙蓉姐姐\hfill 美若天仙}
\renewcommand{\@evenfoot}{\@oddfoot}
}
\makeatother    %恢复现场。

%页眉和页脚常用宏变量
%\thepage 页码
%\thechapter 章编号
%\thesection 节编号
%\chaptername 章起始单词名, Chapter   book专有
%\sectionname 节起始单词名, Section
%\leftmark 左标记,在 article 文档类中包含 section 信息,在
%report 和 book 中则包含 chapter 信息。
%\rightmark 右标记,在 article 中包含 subsection 信息,在
%report 和 book 中则包含 section 信息。



%页面样式
% LATEX 页面样式
%empty 页眉、页脚空白
%plain 页眉空白,页脚含居中页码,非 book 文档类缺省值
%headings 页脚空白,页眉含章节名和页码, book 文档类缺省值
%myheadings 页脚空白,页眉含页码和用户自定义信息
\pagestyle{permanentdamagedhead}  %设置整个文档样式,这里引用了自定义的样式...单面文档奇偶页样式一样,所以需要且只需要定义奇数页的页眉和页脚,偶数页的定义不起作用。
飞

\newpage
\thispagestyle{plain} %单独某页的样式
使用默认的plain样式

\newpage
\markboth{光明左使}{光明右使}
\thispagestyle{myheadings}
使用myheadings,并定义左右标记

\newpage
\thispagestyle{fancy}
\lhead{左擎苍}
\chead{三个代表}
\rhead{右牵黄}
\lfoot{左青龙}
\cfoot{八荣八耻}
\rfoot{右白虎}
\renewcommand{\headrulewidth }{0.4pt}
\renewcommand{\footrulewidth }{0.4pt}
fancy样式

\newpage
\thispagestyle{fancy}
\fancyhf{}
\fancyhead[LE,RO]{\thepage}
\fancyhead[RE]{\leftmark}
\fancyhead[LO]{\rightmark}
\fancypagestyle{plain}{
    \fancyhf{}
    \renewcommand{\headrulewidth}{0pt}
}
\renewcommand\chaptermark [1]{\markboth{\chaptername\ \thechapter: #1}{}}
\renewcommand\sectionmark [1]{\ markright{\thesection: #1}}

fancy宏包,定制章节标记



\end{document}