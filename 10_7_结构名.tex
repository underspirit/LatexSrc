\documentclass[12pt]{article}
\usepackage[CJKchecksingle,CJKnumber]{xeCJK}
\setCJKmainfont[BoldFont={Adobe Heiti Std},ItalicFont={Adobe Kaiti Std}]{Adobe Song Std}
\setCJKsansfont{Adobe Heiti Std}
\setCJKmonofont{Adobe Fangsong Std}
\punctstyle{hangmobanjiao}

%生成书签方式一,colorlinks去掉红框
\usepackage[bookmarksnumbered, bookmarksopen, colorlinks, citecolor=blue, linkcolor=blue]{hyperref}
%好像这样就行,目录有红框
%\usepackage{hyperref}

%方式二,传说中推荐的
%\documentclass[hyperref, UTF8]{ctexart}
%\begin{document}
%\section{中文书签不会乱码}
%UTF-8 编码,Xe\LaTeX{}/pdf\LaTeX{}/\LaTeX{} - DVIPDFMx 编译。
%\end{document}

%方式三
%\documentclass{article}
%\usepackage[hyperref, UTF8]{ctex}
%\begin{document}
%\section{中文书签不会乱码}
%UTF-8 编码,Xe\LaTeX{}/pdf\LaTeX{}/\LaTeX{} - DVIPDFMx 编译。
%\end{document}

%如果我们想改变这些变量的值,比如中文文档需要中文结构名,可以用
%例 10.12 中的方法来重定义这些结构名变量。例中第四、五行代码是为了顺
%应中文的习惯。
\renewcommand{\contentsname}{目录}
\renewcommand{\listfigurename}{图目录}
\renewcommand{\listtablename}{表目录}
\renewcommand{\partname}{第 \thepart 部}
%\renewcommand{\chaptername}{第 \thechapter 章}   %%book专有
\renewcommand{\figurename}{图}
\renewcommand{\tablename}{表}
%\renewcommand{\bibname}{参考文献}   %book专有
\renewcommand{\appendixname}{附录}
\renewcommand{\indexname}{索引}
\renewcommand{\abstractname}{摘要}    %report和article专有
\renewcommand{\refname}{参考文献}   %report和article专有

% \ref 命令显示的是数字。 hyperref 宏包提供了一个
%\autoref 命令,它可以自动判断标签所属结构对象的类型,为引用加上合
%适的名字,输出时显示结构名加上结构编号。该宏包也为此定义了一些结构
%变量名,我们也可以用同样的方法重定义它们
\renewcommand{\equationautorefname}{公式}
\renewcommand{\footnoteautorefname}{脚注}
\renewcommand{\itemautorefname}{项}
\renewcommand{\figureautorefname}{图}
\renewcommand{\tableautorefname}{表}
\renewcommand{\appendixautorefname}{附录}
\renewcommand{\theoremautorefname}{定理}
%需要注意的是, \autoref 命令输出的结果总是名称在编号前面,对于
%章、节等结构无法产生“第 x 章”、“第 x 节”等符合中文习惯的结果。所
%以 例 10.13 略去了若干这样的结构名,我们在引用时需要手工在 \ref 命令
%前后加上合适的字眼。
\begin{document}
    \tableofcontents
    \listoftables
    \listoffigures

    \section{中文书签不会乱码}
    UTF-8 编码,Xe\LaTeX{}/pdf\LaTeX{}/\LaTeX{} - DVIPDFMx 编译。




\end{document}
