\documentclass[12pt]{article}
\usepackage[CJKchecksingle,CJKnumber]{xeCJK}
\setCJKmainfont[BoldFont={Adobe Heiti Std},ItalicFont={Adobe Kaiti Std}]{Adobe Song Std}
\setCJKsansfont{Adobe Heiti Std}
\setCJKmonofont{Adobe Fangsong Std}
\punctstyle{hangmobanjiao}

\usepackage{array}
\usepackage{booktabs}
\usepackage{tabularx}
\usepackage{multirow}
\usepackage{warpcol}
\begin{document}
    \section{section 1}
        %跨列
        %可以使用 \multicolumn 命令,同时使用 booktabs 宏包的 \cmidrule 命令来画横跨几列的横线。它们的语法如下:
        %语法:\multicolumn{横跨列数}{对齐方式}{内容}     语法:\cmidrule{起始列-结束列}
        \begin{table}[htbp]
            \centering
            \caption{跨列}
            \begin{tabular}{lll}
                \toprule
                \ & %第一列,空格
                \multicolumn{2}{c}{常用工具} \\ %跨两列,居中,要换行,否则不对
                \cmidrule{2-3}  %画2-3列的横线
                操作系统 & 发行版 & 编辑器 \\
                \midrule
                Windows & MikTeX & TexMakerX \\
                Unix/Linux & teTeX & Kile \\
                Mac OS & MacTeX & TeXShop \\
                通用 & TeX Live & TeXworks \\
                \bottomrule
            \end{tabular}
        \end{table}

        %跨行
        %跨行表格可以使用 multirow 宏包的 \multirow 命令,其语法如下,
        %语法:\multirow{竖跨行数}{宽度}{内容}
        \begin{table}[htbp]
            \centering
            \begin{tabular}{lll}
                \toprule
                操作系统 & 发行版 & 编辑器 \\
                \midrule
                \multirow{2}{*}{Window}  %后面不用换行,否则不对
                & MikTeX & TexMakerX \\   %还是要加入空白的列
                & teTeX & Kile \\
                Mac OS & MacTeX & TeXShop \\
                通用 & TeX Live & TeXworks \\
                \bottomrule
            \end{tabular}
        \end{table} 
        
        \begin{table}[htbp]
            \centering
            \begin{tabular}{lllc}
                \toprule
                操作系统 & 发行版 & 编辑器 & 用户体验\\
                \midrule
                Windows & MikTeX & TeXnicCenter &
                \multirow {3}{*}{\centering 爽} \\
                Unix/Linux & TeX Live & Emacs \\
                Mac OS & MacTeX & TeXShop \\
                \bottomrule
            \end{tabular}
        \end{table}
       %数字表格
       %当表格中包含大量数字时,手工调整小数点和数位的对齐很麻烦,这时可以使用 Rochester 2 的 warpcol 宏包 [4]。它为 tabular 环境提供了一个列对齐参数 P,其语法如下,其中 m 和 n 分别是小数点前后的位数,数字前的负号可选。
       %语法:P{-m.n}    
        \begin{table}[htbp]
            \centering
            \begin{tabular}{P{2.5}P{-2.5}}  %符号表示最前面留出一个放负号的位置
                \toprule
                %这里必须要使用multicolumn,否则不显示表头,使用 multicolumn 命令是为了保护表头,防止它们被 P 参数误伤。把跨列命令的列数设为 1 是设置单元格格式的一种常用方法。
                \multicolumn{1}{c}{数学常数} &  
                \multicolumn{1}{c}{物理常数} \\
                \midrule
                8 3.14159 & 2.99792 \\
                27.18281 & -17.58819 \\
                \bottomrule
            \end{tabular}
        \end{table}
\end{document}
