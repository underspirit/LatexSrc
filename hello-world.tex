%hello world
\documentclass[10pt,a4paper,notitlepage,onecolumn]{article}
\usepackage{setspace}   %行距命令
\usepackage{marginnote} %不带浮动边注
\usepackage{paralist}   %行距小的列表
\usepackage{verbatim}   %使用大片注释
\title{First Latex}
\author{Lee}
\date{2015/06/14}
\setlength{\parindent}{4em} %设置首行缩进距离
\addtolength{\parskip}{3pt} %设置段间距
\linespread{1}    %行间距
\begin{document}
\begin{comment}
\maketitle  %生成标题
\end{comment}
\setcounter{tocdepth}{4}    %指定目录生成的深度
%\tableofcontents    %生成目录
    \begin{abstract}
        Face detection abstract.
    \end{abstract}
    \part{part 1}
        \begin{spacing}{1.2}    %指定大小行距
            \begin{quotation}
            part 1 content \\ sdf
            \end{quotation}
        \end{spacing}
        \section{section 1}
        \renewcommand{\labelitemi}{*}   %自定义列表符号
        \renewcommand{\theenumi}{\Roman{enumi}}
        \begin{itemize} %各种列表
            \item C++
            \item Java
            \item Python
        \end{itemize}
        \begin{enumerate}
            \item C++
            \item Java
            \item Python
        \end{enumerate}
        \begin{description}
            \item[C++] C++
            \item[Java] Java
            \item[Python] Python
        \end{description}
        \begin{compactitem}
            \item C++
            \item Java
            \item Python
        \end{compactitem}
        section 1 \\ content \\ %换行指令 \\
        %原文打印
        \begin{verbatim}  System.out{"Hello,world!");
         \end{verbatim}
         \renewcommand{\thefootnote}{\Roman{footnote}} %i, ii, iii,自定义脚注标号
         xitele\footnote{asdfsdfdsfdsfdsf}  %脚注
        \# \footnote{xixi}
        \$ \reversemarginpar\marginnote{caca} %先反向,再加边注
         \^ \& \_\_ \{ \} \~ \textbackslash \% - \today -- \TeX --- \LaTeX \\
        \textcircled{A} %各种符号
            \subsection{subsection 1}
            \tiny   %字号大小
            \textrm{subsection 1 content}
            \subsection{subsection 2}
            \huge
            \textbf{subsection 2 content}   %设置加粗
            subsection 2 \newline
            \underline{con} %下划线
            tentsubsection 2 contentsubsection 2 contentsubsection 2 contentsubsection 2 contentsubsection 2 content
                \subsubsection*{subsubsection 1}    % * 不显示标题
                \textsf{subsubsection 1 content}
                    \paragraph{paragraph 1}
                    \textit{paragraph 1 content}    %斜体
                    \\
                        \subparagraph{subparagraph}
                        \textsc{subparagraph 1 content}
\end{document}
