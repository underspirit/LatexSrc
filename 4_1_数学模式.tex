\documentclass[12pt]{article}
\usepackage{amsmath}    %使用数学功能
%行间 (inline) 模式和独立 (display) 模式。前者是指在正文中插入数学内容;后者独立排列,可以有或没有编号
%行间公式和无编号独立公式都有多种输入方法,新手也许会看花了眼。懒人包老师的秘诀是用最短的:行间公式用 $...$,无编号独立公式用\[...\]。建议不要用 $$...$$,因为它和 AMS-L ATEX 有冲突。 amsmath 版本的 equation 环境可以嵌入次环境
\begin{document}
    \section{section 1}
    
    %tex和latex中的行间公式
    Einstein's $E=mc^2$ \ \(E=mc^2\)    
    %行间公式Latex环境
    \begin{math}
        E=mc^2
    \end{math}
    
    %tex和latex中的无编号独立公式
    \[E=mc^2\] $$E=mc^2$$ %
    %无编号独立公式latex环境
    \begin{displaymath}
        E=mc^2
    \end{displaymath}
    %无编号独立公式amsmath环境
    \begin{equation*}   
        E=mc^2
    \end{equation*}
    %
    %有编号公式latex和amsmath环境
    \begin{equation} 
        \boxed{E=mc^2}  %给公式加边框
    \end{equation}
     
\end{document}
