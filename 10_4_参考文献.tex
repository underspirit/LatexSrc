\documentclass[12pt]{article}
\usepackage[CJKchecksingle,CJKnumber]{xeCJK}
\setCJKmainfont[BoldFont={Adobe Heiti Std},ItalicFont={Adobe Kaiti Std}]{Adobe Song Std}
\setCJKsansfont{Adobe Heiti Std}
\setCJKmonofont{Adobe Fangsong Std}
\punctstyle{hangmobanjiao}
\renewcommand\refname{参考文献} %参考文献标题
%如果文档类是book之类的, 用\renewcommand\bibname{参考文献}
\usepackage{amsmath}    %使用数学功能
\usepackage{natbib}
\begin{document}
    %定义了参考文献之后,们我可以用 \cite 命令在正文中引用条目。
    哈利波特$^{\cite{Rowling_1997}}$

    %thebibliography,  LATEX 中最原始的方法是用 thebibliography 环境和 \bibtem 命令来定义参考文献条目。
%    \begin{thebibliography}{99}
%        \bibitem{Rowling_1997}
%        Joanne K. Rowling ,
%        \emph{Harry Potter and the Philosopher 's Stone}.
%        Bloomsbury , London ,
%        1997.
%
%        \bibitem[sb]{Rowl2ing_1997} %指定引用标记
%        Joanne K. Rowling ,
%        \emph{Harry Potter and the Philosopher 's Stone}.
%        Bloomsbury , London ,
%        1997.
%
%        \bibitem{Rowld2ing_1997}
%        Joanne K. Rowling ,
%        \emph{Harry Potter and the Philosopher 's Stone}.
%        Bloomsbury , London ,
%        1997.
%    \end{thebibliography}

    Bibtex \\

    盛世嫡妃\cite{Rowling_1997}

    %BibTex,thebibliography 环境的一个缺点是,用户得自己调整显示格式,这样做很麻烦而且易出错。用数据库文件 .bib 记录参考文献条目,用样式文件 .bst 设置显示格式。普通用户一般不需要改动样式文件,只须维护数据库。
    %BibTEX 将参考文献分为十几种类型,每种类型的参考文献有不同的必选项和可选项

    %一个Bibtex数据实例
    %其中每行是一个数据项,第一个数据项是关键字,供引用时用;其他数据项都以名称 = 值的形式成对出现,值要写在双引号之内;数据项之间用逗号分隔。
%    @book{Rowling_1997,
%        author = "Joanne K. Rowling",
%        title = "Harry Potter and the Sorcerer 's Stone",
%        publisher = "Bloomsbury , London",
%        year = "1997"
%    }

    %有了数据后,我们需要选一个样式。通常的 L ATEX 发行版都会带有四种标准的样式
%    plain 参考文献列表按 作者姓氏 排序,序号为阿拉伯数字。
%    unsrt 参考文献列表按正文中 引用顺序 排序,序号为阿拉伯数字。
%    alpha 参考文献列表按 作者姓氏 排序,序号为作者姓氏加年份。
%    abbrv 类似 plain 样式,作者名字、月份、期刊名等用缩写。
    %选定样式后,我们需要在文档中用 \bibliographystyle 命令来设置样式,然后用 \bibliography 命令输出参考文献列表。
    \bibliographystyle{plain}
    \bibliography{myref}
    %natbib
    % LATEX 提供的 \cite 命令只支持数字模式,natbib 宏包[6] 则同时支持这两种模式。
    %natbib 提供了三种列表样式: plainnat, abbrvnat, unsrtnat,它们的参考文献列表和相对应的 L ATEX 标准样式 plain, abbrv, unsrt 效果相同,只是在引用时可以自由选择作者 -年份或数字模式。

    %设置这些没有,具体用法http://blog.sina.com.cn/s/blog_5e16f1770100lqh2.html  https://zhaoshiliang.wordpress.com/2010/07/27/%E5%8F%82%E8%80%83%E6%96%87%E7%8C%AEnatbib%E7%9A%84%E4%BD%BF%E7%94%A8/
    
    \setcitestyle{authoryear}
    see \cite{Rowling_1997}\\
    see \citet{Rowling_1997}\\
    see \citep{Rowling_1997}

    \setcitestyle{numbers}
    see \cite{Rowling_1997}\\
    see \citet{Rowling_1997}\\
    see \citep{Rowling_1997}

    \setcitestyle{super}
    see \cite{Rowling_1997}\\
    see \citet{Rowling_1997}\\
    see \citep{Rowling_1997}


\end{document}
