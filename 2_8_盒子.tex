\documentclass[12pt]{article}
\begin{document}
    \section{section 1}
        %初级盒子
        \mbox{504 454 4564}\\
        \fbox{504 454 4564} 
        
        %中级盒子
        %语法:[宽度][对齐方式]{内容}  对齐方式有居中 (缺省) 、居左、居右和分散对齐,分别用 c, l, r, s 来表示。
        \makebox[100pt][c]{sbbbbb}\\
        \framebox[100pt][s]{aaaa}
        \\
        %高级盒子
        %大一些的对象比如整个段落可以用 \parbox 命令或 minipage 环境,两者语法类似,有宽度、高度、外部对齐、内部对齐等选项。这里的外部对齐是指该盒子与周围对象的纵向关系,有三种方式:居顶、居中和居底对齐,分别用 t, c, b 来表示。内部对齐是指该盒子内部内容的纵向排列方式,也是同样三种。
        %语法:[外部对齐][高度][内部对齐]{宽度}{内容}
        \fbox{%
         \parbox[c][36pt][t]{170pt}{
        asdfsdfaaaaaaaaaaaaaaaaa
         }%
        }
        \fbox{%
         \begin{minipage}[c][36pt][b]{170pt}
        fdgdfgfggggggggggggggggggg
        \end{minipage}%
        }
\end{document}
