\documentclass[12pt]{article}
\usepackage{marginnote} %不带浮动边注
\usepackage{verbatim}
\begin{document}
        \section{section 1}
            %摘录
             \begin{quote}  %引文两端\\都缩进
                part 1 content \\ sdf
             \end{quote}           
             \begin{quotation}
                part 1 content \\ sdf   %引文两端缩进,首行增加缩进
             \end{quotation}   
             \begin{verse}
                part 1 content \\ sdf   %引文两端缩进,第二行起增加缩进。
             \end{verse}   
            
            %原文打印 文档中的命令和源代码通常使用等宽字体,也就是原文打印
            %使用listing和fancyvrb宏包可以有行号,高亮等..
             \verb|command| %行间命令
              %环境
             \begin{verbatim}  
                 printf("Hello , world!");
             \end{verbatim}
             %带*号,打出空格
             \begin{verbatim*}
                printf("Hello , world!");
             \end{verbatim*}
             
             %脚注,边注 
             %\marginpar 命令使用浮动体 (ìoat) 7 来生成边注,所以不能在其他浮动体或脚注内嵌套。 marginnote 宏包的 \marginnote 命令不使用浮动体,因而没有这个缺陷。
            \renewcommand{\thefootnote}{\Roman{footnote}} %i, ii, iii,自定义脚注标号
             xitele\footnote{asdfsdfdsfdsfdsf}  %脚注
            \# \footnote{xixi}
            \$ \reversemarginpar \marginnote{caca} %先反向,再加边注
            \\
            \normalmarginpar
            ga \marginpar{shabi}
            
            %注释 使用 verbatim 宏包的 comment 环境
            \begin{comment}
            水电费第三方
            水电费
            \end{comment}
\end{document} 
