\documentclass[12pt]{article}
\usepackage{amsmath}    %使用数学功能
\usepackage{amsthm} %使用proof环境
\usepackage[CJKchecksingle,CJKnumber]{xeCJK}
\setCJKmainfont[BoldFont={Adobe Heiti Std},ItalicFont={Adobe Kaiti Std}]{Adobe Song Std}
\setCJKsansfont{Adobe Heiti Std}
\setCJKmonofont{Adobe Fangsong Std}
\punctstyle{hangmobanjiao}
\begin{document}
    \section{section 1}
    \subsection{sub 1}
    %\newtheorem 命令可以用来定义定理之类的环境,其语法如下。语法:{环境名}[编号延续]{显示名}[编号层次]
    %下面的代码定制了四个环境:定义、定理、引理和推论,它们都在一个section 内统一编号,而引理和推论会延续定理的编号。
    \newtheorem{definition}{定义}[section]
    \newtheorem{theorem}{定理}[section]
    \newtheorem{lemma}[theorem]{引理}
    \newtheorem{corollary}[theorem]{推论}
    \begin{definition}
        Java 是一种跨平台的编程语言。
    \end{definition}
    \begin{theorem}
        咖啡因可以刺激人的中枢神经。
    \end{theorem}
    \begin{lemma}
        茶和咖啡都会使人的大脑兴奋。
    \end{lemma}
    \begin{corollary}
        晚上喝咖啡可能会导致失眠。
    \end{corollary}
    
    %amsthm 宏包提供的 proof 环境(见 例 4.21)可以用来输入证明,它会在证明结尾加一个 QED 符号
    \begin{proof}[命题物质无限可分的证明]
        一尺之棰,日取其半,万世不竭。
    \end{proof}
\end{document}
