\documentclass[12pt]{article}
\usepackage[CJKchecksingle,CJKnumber]{xeCJK}
\setCJKmainfont[BoldFont={Adobe Heiti Std},ItalicFont={Adobe Kaiti Std}]{Adobe Song Std}
\setCJKsansfont{Adobe Heiti Std}
\setCJKmonofont{Adobe Fangsong Std}
\punctstyle{hangmobanjiao}

\usepackage{array}
\usepackage{booktabs}
\usepackage{tabularx}
\begin{document}
    \section{section 1}
        %鉴定表格
        %用 \hline 命令表示横线, |表示竖线;用 & 来分列,用 \\ 来换行;每列可以采用居中、居左、居右等横向对齐方式,分别用 l、 c、 r 来表示。
        \begin{tabular}{|l|c|r|}    %指明列数和每列的对齐方式,以及竖线 语法:[纵向对齐]{横向对齐和分隔符}
            \hline
            操作系统 & 发行版 & 编辑器 \\
            \hline
            Windows & MikTeX & TexMakerX \\
            \hline
            Unix/Linux & teTeX & Kile \\
            \hline
            Mac OS & MacTeX & TeXShop \\
            \hline
             通用 & TeX Live & TeXworks \\
            \hline
    \end{tabular}

    %table,使用浮动表格环境(使用与figure环境类似),并且使用三线表宏包booktabs
    \begin{table}[htbp]
        \centering
        \begin{tabular}{lll}
            \toprule
            操作系统 & 发行版 & 编辑器 \\
            \midrule
            Windows & MikTeX & TexMakerX \\
            Unix/Linux & teTeX & Kile \\
            Mac OS & MacTeX & TeXShop \\
            通用 & TeX Live & TeXworks \\
            \bottomrule
        \end{tabular}
    \end{table}
    
    %宽度控制,控制某列的宽度,可以将其对齐方式参数从 l、 c、 r 改为p{宽度}。这时纵向对齐方式是居顶, t、 c、 b 等参数失效。
    \begin{table}[htbp]
        \centering
        \begin{tabular}{p{80pt}p{80pt}p{70pt}}
            \toprule
            操作系统 & 发行版 & 编辑器 \\
            \midrule
            Windows & MikTeX & TexMakerX \\
            Unix/Linux & teTeX & Kile \\
            Mac OS & MacTeX & TeXShop \\
            通用 & TeX Live & TeXworks \\
            \bottomrule
        \end{tabular}
    \end{table}
    
    %使用宽度控制参数之后,表格内容缺省居左对齐。我们可以用列前置命令 >{} 配合 \centering、 \raggedleft 命令来把横向对齐方式改成居中或居右。列前置命令仅对紧邻其后的一列有效,其语法如下:语法:>{命令}列参数
    \begin{table}[htbp]
        \centering
        \begin{tabular}{p{80pt}>{\centering}p{80pt}>{\raggedleft\arraybackslash}p{80pt}} %使用array宏包
            \toprule
            操作系统 & 发行版 & 编辑器 \\
            \midrule
            Windows & MikTeX & TexMakerX \\
            Unix/Linux & teTeX & Kile \\
            Mac OS & MacTeX & TeXShop \\
            通用 & TeX Live & TeXworks \\
            \bottomrule
        \end{tabular}
    \end{table}
    
    %如果想把纵向对齐方式改为居中和居底,可以使用 Mittelbach和 Carlisle的 array 宏包[3],它提供了另两个对齐方式参数: m{宽度}、 b{宽度}。 
    
    %控制整个表格的宽度
    %使用 Carlisle的 tabularx 宏包 [2] 的同名环境,其语法如下,其中 X 参数表示某列可以折行。语法:{表格宽度}{横向对齐、分隔符、折行}
    \begin{table}[htbp]
        \centering
        \begin{tabularx}{350pt}{l||X|l||X}
            \toprule
            李白 & 平林漠漠烟如织,寒山一带伤心碧。暝色入高楼,有人楼上愁。
            玉阶空伫立,宿鸟归飞急。何处是归程,长亭更短亭。 &
            泰戈尔 & 夏天的飞鸟,飞到我的窗前唱歌,又飞去了。
            秋天的黄叶,它们没有什么可唱,只叹息一声,飞落在那里。 \\
            \bottomrule
        \end{tabularx}
    \end{table}
    
\end{document}
