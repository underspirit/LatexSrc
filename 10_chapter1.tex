%彩色表格
%使用 Carlisle 的 colortbl 宏包[7]。它提供的 \columncolor、 \rowcolor、 \cellcolor 命令可以分别设置列、行、单元格的颜色。这三个命令的基本语法相似:
%语法:{颜色}
%\columncolor 需要放到列前置命令里, rowcolor、 \cellcolor 分别放到行、单元格之前。 colortbl 宏包可以使用 xcolor 宏包的色彩模型;两者同时,前者不能直接加载,需要通过后者的选项 table 来加载。三个命令同时使用时,它们的优先顺序为:单元格、行、列。
\begin{table}[htbp]
    \centering
    \caption{彩色表格}
    \begin{tabular}{l>{\columncolor{Yellow}}ll}
        \rowcolor{Red}操作系统 & 发行版 & 编辑器 \\
        Windows & MikTeX & TexMakerX \\
        \rowcolor{Green}Unix/Linux & \cellcolor{Lavender}teTeX
        & Kile \\
        Mac OS & MacTeX & TeXShop \\
        \rowcolor{Blue}通用 & TeX Live & TeXworks \\
    \end{tabular}
\end{table}

%奇偶行颜色,xcolor 宏包的 rowcolors 命令 (需要 colortbl 宏包的支持) 可以分别设置奇偶行的颜色,甚合吾意。该命令语法如下:语法:{起始行}{奇数行颜色}{偶数行颜色}
\begin{table}[htbp]
    \centering
    \caption{奇偶行彩色表格}
    \rowcolors{1}{White}{Lavender}
    \begin{tabular}{lll}
        \hline
        操作系统 & 发行版 & 编辑器 \\
        Windows & MikTeX & TexMakerX \\
        Unix/Linux & teTeX & Kile \\
        Mac OS & MacTeX & TeXShop \\
        通用 & TeX Live & TeXworks \\
        \hline
        \hiderowcolors  %暂停显示前面设置的奇偶行颜色,否则后面的其他表格会继续显示颜色,\showrowcolors 可以用来重新激活奇偶行颜色设置。
    \end{tabular}
\end{table}