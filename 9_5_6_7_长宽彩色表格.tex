\documentclass[12pt]{article}
\usepackage[CJKchecksingle,CJKnumber]{xeCJK}
\setCJKmainfont[BoldFont={Adobe Heiti Std},ItalicFont={Adobe Kaiti Std}]{Adobe Song Std}
\setCJKsansfont{Adobe Heiti Std}
\setCJKmonofont{Adobe Fangsong Std}
\punctstyle{hangmobanjiao}

\usepackage{booktabs}
\usepackage{longtable}
\usepackage{rotating}
\usepackage{tabularx}
\usepackage[table,dvipsnames]{xcolor}
\begin{document}
    \section{section 1}
        %长表格
        %表格太长要跨页,可以使用 Carlisle的 longtable 宏包
        %1. 首先用 longtable 环境取代 tabular 环境;
        %2. 然后在表格开始部分定义每页页首出现的通用表头,表头最后一行末尾不用 \\ 换行,而是加一个 \endhead 命令;
        %3. 接着定义首页表头 (如果它和通用表头不同的话) ,同样地最后一行用\endfirsthead 命令结尾;
        %4. 然后是以 \endfoot 命令结尾的通用表尾;
        %5. 然后是以 \endlastfoot 命令结尾的末页表尾 (如果它和通用表尾不同的话) ;
        %6. 最后是表格的具体内容。
        \begin{longtable}{ll}
            \multicolumn{2}{r}{接上页} \\  %在toprule之前
            \toprule
            作者 & 作品 \\
            \midrule
            \endhead
            \caption{长表格} \\    %标注要放在这里...
            \toprule
            作者 & 文字 \\
            \midrule
            \endfirsthead
            \bottomrule
            \multicolumn{2}{r}{接下页\dots} \\     %%在bottomrule之后
            \endfoot
            \bottomrule
            \endlastfoot
            白居易 & 汉皇重色思倾国,御宇多年求不得。 \\
            & 杨家有女初长成,养在深闺人未识。 \\
            & 天生丽质难自弃,一朝选在君王侧。 \\
            & 回眸一笑百媚生,六宫粉黛无颜色。 \\
            & 春寒赐浴华清池,温泉水滑洗凝脂。 \\
            & 侍儿扶起娇无力,始是新承恩泽时。 \\
            & 云鬓花颜金步摇,芙蓉帐暖度春宵。 \\
            & 春宵苦短日高起,从此君王不早朝。 \\
            & 承欢侍宴无闲暇,春从春游夜专夜。 \\
            & 后宫佳丽三千人,三千宠爱在一身。 \\
            & 金屋妆成娇侍夜,玉楼宴罢醉和春。 \\
            & 姊妹弟兄皆列土,可怜光彩生门户。 \\
            & 遂令天下父母心,不重生男重生女。 \\
            & 骊宫高处入青云,仙乐风飘处处闻。 \\
            & 缓歌慢舞凝丝竹,尽日君王看不足。 \\
            & 渔阳鼙鼓动地来,惊破霓裳羽衣曲。 \\
            & 九重城阙烟尘生,千乘万骑西南行。 \\
            & 翠华摇摇行复止,西出都门百余里。 \\
            & 六军不发无奈何,宛转蛾眉马前死。 \\
            & 花钿委地无人收,翠翅金雀玉搔头。 \\
            & 君王掩面救不得,回看血泪相和流。 \\
            & 黄埃散漫风萧索,云栈萦纡登剑阁。 \\
            & 峨嵋山下少人行,旌旗无光日色薄。 \\
            & 蜀江水碧蜀山青,圣主朝朝暮暮情。 \\
            & 行宫见月伤心色,夜雨闻铃断肠声。 \\
            & 杨家有女初长成,养在深闺人未识。 \\
            & 天生丽质难自弃,一朝选在君王侧。 \\
            & 回眸一笑百媚生,六宫粉黛无颜色。 \\
            & 春寒赐浴华清池,温泉水滑洗凝脂。 \\
            & 侍儿扶起娇无力,始是新承恩泽时。 \\
            & 云鬓花颜金步摇,芙蓉帐暖度春宵。 \\
            & 春宵苦短日高起,从此君王不早朝。 \\
            & 承欢侍宴无闲暇,春从春游夜专夜。 \\
            & 后宫佳丽三千人,三千宠爱在一身。 \\
            & 金屋妆成娇侍夜,玉楼宴罢醉和春。 \\
            & 姊妹弟兄皆列土,可怜光彩生门户。 \\
            & 遂令天下父母心,不重生男重生女。 \\
            & 骊宫高处入青云,仙乐风飘处处闻。 \\
            & 缓歌慢舞凝丝竹,尽日君王看不足。 \\
            & 渔阳鼙鼓动地来,惊破霓裳羽衣曲。 \\
            & 九重城阙烟尘生,千乘万骑西南行。 \\
            & 翠华摇摇行复止,西出都门百余里。 \\
            & 六军不发无奈何,宛转蛾眉马前死。 \\
            & 花钿委地无人收,翠翅金雀玉搔头。 \\
            & 君王掩面救不得,回看血泪相和流。 \\
            & 黄埃散漫风萧索,云栈萦纡登剑阁。 \\
            & 峨嵋山下少人行,旌旗无光日色薄。 \\
            & 蜀江水碧蜀山青,圣主朝朝暮暮情。 \\
            & 行宫见月伤心色,夜雨闻铃断肠声。 \\
        \end{longtable}

        %宽表格
        %表格太宽时可以使用 Fairbairns 3 等人的 rotating 宏包 [6]。其方法很简单,用 sidewaystable 环境替代 table 环境即可。
        \begin{sidewaystable}[htbp]
            \caption{主流英文词典}
            \label{tab:dict}
            \centering
            \begin{tabularx}{550pt}{Xllcrrr}
                \toprule
                Title & Abbr & Publisher & Year & Pages & Entries &
                Price \\
                \midrule
                Oxford English Dict , 2nd Ed & OED & Oxford Univ
                & 1989 & 21,728 & 616,500 & 995 \\
                \midrule
                Shorter Oxford English Dict , 7th Ed & SOED & Oxford
                Univ
                & 2007 & 3,888 & 600,000 & 175 \\
                New Oxford Dict of English , 2nd & NODE & Oxford Univ
                & 2005 & 2,112 & 355,000 & 68 \\
                Webster 's Third New International Dict & W3 & Merriam -
                Webster
                & 1961 & 2,816 & 476,000 & 129 \\
                American Heritage Dict , 4th Ed & AHD & Houghton
                Mifflin
                & 2000 & 2,112 & 90,000 & 60 \\
                Random House Webster 's Unabridged Dict , 2nd Ed &
                Random & Random House
                & 2005 & 2,256 & 315,000 & 69 \\
                \midrule
                Concise Oxford Dict , 11th Ed & COD & Oxford Univ
                & 2006 & 1,728 & 240,000 & \\
                Chambers Dict , 10th Ed & Chambers & Chambers Harrap
                & 2006 & 1,872 & & 50 \\
                Collins English Dict , 9th Ed & Collins & HarperCollins
                & 2007 & 1,888 & & 67 \\
                Longman Dict of Contemporary English , 4th Ed & Longman
                & Longman
                & 2005 & & 207,000 & 71 \\
                Merriam -Webster 's Collegiate Dict , 11th Ed & & Merriam
                -Webster
                & 2003 & 1,664 & 225,000 & 26 \\
                American Heritage College Dict , 4th Ed & & Houghton
                Mifflin
                & 2007 & 1,664 & & 26 \\
                Random House Webster 's College Dict & & Random House
                & 2005 & 1,632 & & 26 \\
                Webster 's New World College Dict , 4th Ed & & John
                Wiley \& Sons
                & 2004 & 1,744 & 160,000 & 26 \\
                \bottomrule
            \end{tabularx}
        \end{sidewaystable}
        
        %彩色表格
        %使用 Carlisle 的 colortbl 宏包[7]。它提供的 \columncolor、 \rowcolor、 \cellcolor 命令可以分别设置列、行、单元格的颜色。这三个命令的基本语法相似:
        %语法:{颜色}
        %\columncolor 需要放到列前置命令里, rowcolor、 \cellcolor 分别放到行、单元格之前。 colortbl 宏包可以使用 xcolor 宏包的色彩模型;两者同时,前者不能直接加载,需要通过后者的选项 table 来加载。三个命令同时使用时,它们的优先顺序为:单元格、行、列。
        \begin{table}[htbp]
            \centering
            \begin{tabular}{l>{\columncolor{Yellow}}ll}
                \rowcolor{Red}操作系统 & 发行版 & 编辑器 \\
                Windows & MikTeX & TexMakerX \\
                \rowcolor{Green}Unix/Linux & \cellcolor{Lavender}teTeX
                & Kile \\
                Mac OS & MacTeX & TeXShop \\
                \rowcolor{Blue}通用 & TeX Live & TeXworks \\
            \end{tabular}
        \end{table}
        
        %奇偶行颜色,xcolor 宏包的 rowcolors 命令 (需要 colortbl 宏包的支持) 可以分别设置奇偶行的颜色,甚合吾意。该命令语法如下:语法:{起始行}{奇数行颜色}{偶数行颜色}
        \begin{table}[htbp]
            \centering
            \rowcolors{1}{White}{Lavender}
            \begin{tabular}{lll}
                \hline
                操作系统 & 发行版 & 编辑器 \\
                Windows & MikTeX & TexMakerX \\
                Unix/Linux & teTeX & Kile \\
                Mac OS & MacTeX & TeXShop \\
                通用 & TeX Live & TeXworks \\
                \hline
                \hiderowcolors  %暂停显示前面设置的奇偶行颜色,否则后面的其他表格会继续显示颜色,\showrowcolors 可以用来重新激活奇偶行颜色设置。
            \end{tabular}
        \end{table}

\end{document}
