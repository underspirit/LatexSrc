\documentclass[12pt]{article}
\usepackage[CJKchecksingle,CJKnumber]{xeCJK}
\setCJKmainfont[BoldFont={Adobe Heiti Std},ItalicFont={Adobe Kaiti Std}]{Adobe Song Std}
\setCJKsansfont{Adobe Heiti Std}
\setCJKmonofont{Adobe Fangsong Std}
\punctstyle{hangmobanjiao}

\usepackage{hyperref}

\usepackage{makeidx}
\makeindex  %必须放在这
\begin{document}
    \section{333}
    \subsection{123}
    %超链接,hyperref宏包
    %\hyperref 命令对已经定义的标签进行简单包装,加上文字描述。
    \label{sec:hyperlink}
    编号形式的链接:\ref{sec:hyperlink}\\
    文字形式的链接:\hyperref[sec:hyperlink]{链接}\\
    \url{http://www.dralpha.com/}\\
    \href{http://www.dralpha.com/}{包老师的主页}\\
    %索引,干什么还不知道
    %    当编译含索引的文档时,用户需要执行三次编译操作,
    %1. 第一遍 xelatex 把索引条目写到一个 .idx 文件中去。
    %2. makeindex 把 .idx 排序后写到一个 .ind 文件中去。
    %3. 第二遍 xelatex 在 \printindex 命令的地方引用 .ind 的内容,生成
    %正确的文档。
    \index{aaaaaa}
    水电费
    \printindex
    \index{ggg}
    水电费
    \printindex
\end{document}