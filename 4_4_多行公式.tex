\documentclass[12pt]{article}
\usepackage{amsmath}    %使用数学功能
\begin{document}
    \section{section 1}
    %长公式
    %无对齐,multline
    \begin{multline}
        x = a+b+c+{} \\
        d+e+f+g
    \end{multline}
    %有对齐,split它本身不能独立使用,必须包含在其它数学环境内,因此也称作次环境。它用 \\ 和 & 来分行和设置对齐的位置。
    \[ \begin{split}
        x ={} &a+b+c+{} \\
              &d+e+f+g
    \end{split} \]
    
    %公式组
    %无对齐,gather
    \begin{gather}
        a = b+c+d \\
        x = y+z
    \end{gather}
    %对齐,multline, gather, align 等环境都有带 * 的版本,不生成公式编号。
    \begin{align*}
        a &= b+c+d \\
        x &= y+z
    \end{align*}
    
    %分支公式,cases
    \[ y=\begin{cases}
        -x,\quad x<0 \\
        0,\quad x=0 \\
        x,\quad x>0
    \end{cases} \]
\end{document}
